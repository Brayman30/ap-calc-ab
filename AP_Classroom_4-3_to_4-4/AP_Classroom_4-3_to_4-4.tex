\documentclass[12pt,letterpaper, onecolumn]{exam}
\usepackage{amsmath}
\usepackage{amssymb}
\usepackage{nicefrac}
\usepackage{graphicx}
\usepackage{tocbasic}
\usepackage{hyperref}

\usepackage{xcolor}
\hypersetup{
	colorlinks=true,
	linkcolor=blue,
	filecolor=magenta,      
	urlcolor=cyan,
}

\usepackage[lmargin=71pt, tmargin=1.2in]{geometry}  %For centering solution box
\lhead{AP Calculus AB -- APC 4.1 -- 4.2 Assignment\\}
\rhead{Brayden Price\\}
% \chead{\hline} % Un-comment to draw line below header
\thispagestyle{empty}   %For removing header/footer from page 1
\newcommand\at[2]{\left.#1\right|_{#2}}


\DeclareNewTOC[%
type=exercise,%
types=exercises,%
name=Exercise,%
listname={List of Questions},%
tocentrystyle=tocline,%
tocentryindent=0pt,%
tocentrydynnumwidth,%
tocentrypagenumberformat=\entryprefix{page~},%
tocentrypagenumberbox=\mbox
]{exr}
\newcommand*\entryprefix[2]{#1#2}

\qformat{%
	\textbf{Question~\thequestion}%
	\addxcontentsline{exr}{exercise}{Question~\thequestion}%
	\hfill\thepoints%
}

\begin{document}
	
	\begingroup  
	
	\LARGE AP Calculus AB\\
	\LARGE AP Classroom: 4.3 -- 4.4\\[0.5em]
	\large \today\\[0.5em]
	\large Brayden Price \\
	All final solutions are \boxed{boxed} \\
	\endgroup
	\rule{\textwidth}{0.4pt}
	
	\listofexercises
	\clearpage
	
	\printanswers
	\renewcommand{\solutiontitle}{\noindent\textbf{Solution:}\enspace}   %Replace "Ans:" with starting keyword in solution box
	
	\begin{questions}
		
		%Q1
		\question
			%BEGIN_FOLD
			\begin{tabular}{l||l|l|l|l}
			 $x$ &  $f(x)$ &  $f^{\prime}(x)$ &  $g(x)$ & $g^{\prime}(x)$ \\ \hline \hline
			 1 &  6    &  4     &  2    & 5     \\
			 2 &  9    &  2     &  3    & 1     \\
			 3 &  10   &  -4    &  4    & 2     \\
			 4 &  -1   &  3     &  6    & 7    
			\end{tabular} \\
			The functions $f$ and $g$ are differentiable for all real numbers, and $g$ is strictly increasing. The table above gives values of the functions and their first derivatives at selected values of $x$. The function $h$ is given by $h(x) = f(g(x)) - 6$. \\
			If $g^{-1}$ is the inverse function of $g$, write an equation for the line tangent to the graph of $y = g^{-1}$ at $x=2$.
		
			\begin{solution}
				Given that $g(1) = 2$, the point of tangency on the graph of $y = g^{-1}$ is the point $(2,1)$. {\footnotesize (since $g^{-1}$ is an inverse of $g$, $g(2)=1$.)} \\
				We now want to find the derivative at the point of tangency: \\ 
				To find $\left[ g^{-1} \right] ^ {\prime} \left( 2 \right)$, we must use the rule for derivatives of inverses:
				\begin{align*}
					\left[ g^{-1} \right] ^ {\prime} \left( x \right) &= \frac { 1 } { g^{\prime} \left( g^{-1} \left( x \right)  \right) } \\
					\left[ g^{-1} \right] ^ {\prime} \left( 2 \right) &= \frac { 1 } { g^{\prime} \left( g^{-1} \left( 2 \right)  \right) } \\
																	  &= \frac { 1 } { g^{\prime} \left( 1 \right) } \\
																	  &= \frac { 1 } { 5 }
				\end{align*}
				Now, we just need to use the point $(2,1)$ and the derivative at that point to write the tangent line: \\
				$$ \boxed{y-1 = \nicefrac{1}{5} \left( x-2 \right)} $$
			\end{solution}
		%END_FOLD
		
		%Q2
		\question A student attempted to confirm that the function $f$ defined by $f(x) = \frac{ x^2 + x - 6}{x^2 -7x + 10}$ is continuous at $x = 2$
			%BEGIN_FOLD
		In which step, if any, does an error first appear?
			\begin{enumerate}
					\item [Step 1:] $$f(x) = \frac {x^2 + x - 6} {x^2 - 7x + 10} = \frac { (x-2) (x+3) } { (x-2) (x-5) }$$
					\item [Step 2:] $$ \lim_{x \to 2} f(x) = \lim_{x \to 2} \frac {x+3}{x-5} = \frac {2+3}{2-5} = - \frac{5}{3}$$
					\item [Step 3:] $$f(2) = \frac {2+3}{2-5} = - \frac{5}{3}$$
					\item [Step 4:] $$ \lim_{x \to 2} f(x) = f(2) \text{, so } f \text{ is continuous at } x=2.$$
			\end{enumerate}
			\begin{solution}
				During Step 3, we made a little bit of a mistake. While in the limit in Step 2, we can simplify the rational definition of $f(x)$, when finding the actual value (or lack thereof) of $f(2)$, we \textbf{\emph{cannot}} do this. \\
				Even though $f$ is actually undefined at $x=2$ (removable discontinuity / hole), we incorrectly said it was equal to $\frac{-5}{3}$, and therefore incorrectly stated $f$ as continuous at $x=2$. 
			\end{solution}
		%END_FOLD
		
		%Q3
		\question Two particles move along the $x$-axis with velocities given by $v_1(t) = -3t^2 + 8t +5$ and $v_2(t) = \sin t$ for time $t \geq 0$. At what time $t$ do the two particles have equal acceleration?
			\begin{solution}
				To find acceleration from give velocity equations, we must take the derivative of the velocity functions to find two new acceleration equations
				\begin{align*}
					a_1(t) &= \frac{d}{dx} \left( v_1(t) \right) \\
							&= \frac{d}{dx} \left( -3t^2 + 8t +5 \right) \\
							&= -6t + 8 \\ \\
					a_2(t)	&= \frac{d}{dx} \left( v_2(t) \right) \\
							&= \frac{d}{dx} \left( \sin t \right) \\
							&= \cos t \\
				\end{align*}
				Now, we need to find the time $t$ where $a_1(t)=a_2(t)$
				\begin{align*}
					a_1(t) 	&= a_2(t) \\
					-6t + 8 &= \cos t
				\end{align*}
				Since this is a calculator required question, just use your calculator to find the intersection of the line and the cosine function. (\textbf{Make sure your calculator is in radian mode.})
				$$\boxed{t\approx 1.2866}$$
			\end{solution}
	\end{questions}
\end{document}